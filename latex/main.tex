\documentclass[french,a4paper]{book} % ou autre classe
\usepackage[utf8]{inputenc}
\usepackage[T1]{fontenc}
\usepackage{tikz}
\usepackage{pgfplots}
\usepackage{caption}
\usepackage{subcaption}
\usepackage{fullpage}

\pgfplotsset{
    step/.style={width=1.\linewidth,height=0.65\linewidth,xmin=0,xlabel near ticks,ylabel near ticks,grid=both,mark=none,xlabel=temps (s)},
    impulse/.style={width=1.\linewidth,height=0.65\linewidth,xmin=0,xlabel near ticks,ylabel near ticks,grid=both,mark=none,xlabel=temps (s)},
    freqresp/.style={width=1.\linewidth,height=0.65\linewidth,xlabel near ticks,ylabel near ticks,grid=both,mark=none,xlabel=Pulsation (rad/s)},
}



\begin{document}

\chapter{LP}
\newpage
\begin{figure}
\centering
	\begin{subfigure}{.5\textwidth}
  	\centering
  	\begin{tikzpicture}
    		\begin{axis}[impulse,xmax=0.04]
    			\addplot+[thick] table[mark=none,x index=0, y index=1, col sep=comma] {./csv/LP_0_impulse.csv}; 
    		\end{axis}
	\end{tikzpicture}
 	 \caption{Réponse Impulsionnelle :  $m=0.1$}
  	\label{fig:sub1}
	\end{subfigure}\hfill
	\begin{subfigure}{.5\textwidth}
  	\centering
  	\begin{tikzpicture}
    		\begin{axis}[impulse,xmax=0.06]
    			\addplot+[thick] table[mark=none,x index=0, y index=1, col sep=comma] {./csv/LP_1_impulse.csv}; 
    		\end{axis}
	\end{tikzpicture}
 	 \caption{Réponse Impulsionnelle : $m=3$}
  	\label{fig:sub1}
	\end{subfigure}
	\vskip 1em
	\begin{subfigure}{.5\textwidth}
  	\centering
  	\begin{tikzpicture}
    		\begin{axis}[step,xmax=0.04]
    			\addplot+[thick] table[mark=none,x index=0, y index=1, col sep=comma] {./csv/LP_0_step.csv}; 
    		\end{axis}
	\end{tikzpicture}
 	 \caption{Réponse Indicielle : $m=0.1$}
  	\label{fig:sub1}
	\end{subfigure}\hfill
	\begin{subfigure}{.5\textwidth}
  	\centering
  	\begin{tikzpicture}
    		\begin{axis}[step,xmax=0.06]
    			\addplot+[thick] table[mark=none,x index=0, y index=1, col sep=comma] {./csv/LP_1_step.csv}; 
    		\end{axis}
	\end{tikzpicture}
 	 \caption{Réponse Indicielle : $m=3$}
  	\label{fig:sub1}
	\end{subfigure}
	\vskip 1em
	\begin{subfigure}{.5\textwidth}
  	\centering
  	\begin{tikzpicture}
    		\begin{loglogaxis}[freqresp,xmin=10,xmax=100000]
    			\addplot+[thick] table[mark=none,x index=0, y index=1, col sep=comma] {./csv/LP_0_freqresp.csv}; 
    		\end{loglogaxis}
	\end{tikzpicture}
 	 \caption{Module de la Réponse Fréquentielle :  $m=0.1$}
  	\label{fig:sub1}
	\end{subfigure}\hfill
	\begin{subfigure}{.5\textwidth}
  	\centering
  	\begin{tikzpicture}
    		\begin{loglogaxis}[freqresp,xmin=100,xmax=10000]
    			\addplot+[thick] table[mark=none,x index=0, y index=1, col sep=comma] {./csv/LP_1_freqresp.csv}; 
    		\end{loglogaxis}
	\end{tikzpicture}
 	 \caption{Module de la Réponse Fréquentielle : $m=3$}
  	\label{fig:sub1}
	\end{subfigure}
	\vskip 1em
	\begin{subfigure}{.5\textwidth}
  	\centering
  	\begin{tikzpicture}
    		\begin{semilogxaxis}[freqresp,xmin=10,xmax=100000]
    			\addplot+[thick] table[mark=none,x index=0, y index=2, col sep=comma] {./csv/LP_0_freqresp.csv}; 
    		\end{semilogxaxis}
	\end{tikzpicture}
 	 \caption{Phase de la Réponse Fréquentielle :  $m=0.1$}
  	\label{fig:sub1}
	\end{subfigure}\hfill
	\begin{subfigure}{.5\textwidth}
  	\centering
  	\begin{tikzpicture}
    		\begin{semilogxaxis}[freqresp,xmin=100,xmax=10000]
    			\addplot+[thick] table[mark=none,x index=0, y index=2, col sep=comma] {./csv/LP_1_freqresp.csv}; 
    		\end{semilogxaxis}
	\end{tikzpicture}
 	 \caption{Phase de la Réponse Fréquentielle :  $m=3$}
  	\label{fig:sub1}
	\end{subfigure}
\caption{Analyse de deux filtres passe-bas de second ordre ($T0=2$, $\omega_0=1000$ rad/s). Les figures de gauche correspondent au cas où $m=0.1$ et les figures de droite au cas où $m=3$.}
\label{fig:test}
\end{figure}



\newpage

\chapter{HP}

\begin{figure}
\centering
	\begin{subfigure}{.5\textwidth}
  	\centering
  	\begin{tikzpicture}
    		\begin{axis}[impulse,xmax=0.04]
    			\addplot+[thick] table[mark=none,x index=0, y index=1, col sep=comma] {./csv/HP_0_impulse.csv}; 
    		\end{axis}
	\end{tikzpicture}
 	 \caption{Réponse Impulsionnelle :  $m=0.1$}
  	\label{fig:sub1}
	\end{subfigure}\hfill
	\begin{subfigure}{.5\textwidth}
  	\centering
  	\begin{tikzpicture}
    		\begin{axis}[impulse,xmax=0.06]
    			\addplot+[thick] table[mark=none,x index=0, y index=1, col sep=comma] {./csv/HP_1_impulse.csv}; 
    		\end{axis}
	\end{tikzpicture}
 	 \caption{Réponse Impulsionnelle : $m=3$}
  	\label{fig:sub1}
	\end{subfigure}
	\vskip 1em
	\begin{subfigure}{.5\textwidth}
  	\centering
  	\begin{tikzpicture}
    		\begin{axis}[step,xmax=0.04]
    			\addplot+[thick] table[mark=none,x index=0, y index=1, col sep=comma] {./csv/HP_0_step.csv}; 
    		\end{axis}
	\end{tikzpicture}
 	 \caption{Réponse Indicielle : $m=0.1$}
  	\label{fig:sub1}
	\end{subfigure}\hfill
	\begin{subfigure}{.5\textwidth}
  	\centering
  	\begin{tikzpicture}
    		\begin{axis}[step,xmax=0.06]
    			\addplot+[thick] table[mark=none,x index=0, y index=1, col sep=comma] {./csv/HP_1_step.csv}; 
    		\end{axis}
	\end{tikzpicture}
 	 \caption{Réponse Indicielle : $m=3$}
  	\label{fig:sub1}
	\end{subfigure}
	\vskip 1em
	\begin{subfigure}{.5\textwidth}
  	\centering
  	\begin{tikzpicture}
    		\begin{loglogaxis}[freqresp,xmin=10,xmax=100000]
    			\addplot+[thick] table[mark=none,x index=0, y index=1, col sep=comma] {./csv/HP_0_freqresp.csv}; 
    		\end{loglogaxis}
	\end{tikzpicture}
 	 \caption{Module de la Réponse Fréquentielle :  $m=0.1$}
  	\label{fig:sub1}
	\end{subfigure}\hfill
	\begin{subfigure}{.5\textwidth}
  	\centering
  	\begin{tikzpicture}
    		\begin{loglogaxis}[freqresp,xmin=100,xmax=10000]
    			\addplot+[thick] table[mark=none,x index=0, y index=1, col sep=comma] {./csv/HP_1_freqresp.csv}; 
    		\end{loglogaxis}
	\end{tikzpicture}
 	 \caption{Module de la Réponse Fréquentielle : $m=3$}
  	\label{fig:sub1}
	\end{subfigure}
	\vskip 1em
	\begin{subfigure}{.5\textwidth}
  	\centering
  	\begin{tikzpicture}
    		\begin{semilogxaxis}[freqresp,xmin=10,xmax=100000]
    			\addplot+[thick] table[mark=none,x index=0, y index=2, col sep=comma] {./csv/HP_0_freqresp.csv}; 
    		\end{semilogxaxis}
	\end{tikzpicture}
 	 \caption{Phase de la Réponse Fréquentielle :  $m=0.1$}
  	\label{fig:sub1}
	\end{subfigure}\hfill
	\begin{subfigure}{.5\textwidth}
  	\centering
  	\begin{tikzpicture}
    		\begin{semilogxaxis}[freqresp,xmin=100,xmax=10000]
    			\addplot+[thick] table[mark=none,x index=0, y index=2, col sep=comma] {./csv/HP_1_freqresp.csv}; 
    		\end{semilogxaxis}
	\end{tikzpicture}
 	 \caption{Phase de la Réponse Fréquentielle :  $m=3$}
  	\label{fig:sub1}
	\end{subfigure}
\caption{Analyse de deux filtres passe-hut de second ordre ($T_{\infty}=2$, $\omega_0=1000$ rad/s). Les figures de gauche correspondent au cas où $m=0.1$ et les figures de droite au cas où $m=3$.}
\label{fig:test}
\end{figure}


\newpage

\chapter{BP}

\begin{figure}
\centering
	\begin{subfigure}{.5\textwidth}
  	\centering
  	\begin{tikzpicture}
    		\begin{axis}[impulse,xmax=0.04]
    			\addplot+[thick] table[mark=none,x index=0, y index=1, col sep=comma] {./csv/BP_0_impulse.csv}; 
    		\end{axis}
	\end{tikzpicture}
 	 \caption{Réponse Impulsionnelle :  $m=0.1$}
  	\label{fig:sub1}
	\end{subfigure}\hfill
	\begin{subfigure}{.5\textwidth}
  	\centering
  	\begin{tikzpicture}
    		\begin{axis}[impulse,xmax=0.06]
    			\addplot+[thick] table[mark=none,x index=0, y index=1, col sep=comma] {./csv/BP_1_impulse.csv}; 
    		\end{axis}
	\end{tikzpicture}
 	 \caption{Réponse Impulsionnelle : $m=3$}
  	\label{fig:sub1}
	\end{subfigure}
	\vskip 1em
	\begin{subfigure}{.5\textwidth}
  	\centering
  	\begin{tikzpicture}
    		\begin{axis}[step,xmax=0.04]
    			\addplot+[thick] table[mark=none,x index=0, y index=1, col sep=comma] {./csv/BP_0_step.csv}; 
    		\end{axis}
	\end{tikzpicture}
 	 \caption{Réponse Indicielle : $m=0.1$}
  	\label{fig:sub1}
	\end{subfigure}\hfill
	\begin{subfigure}{.5\textwidth}
  	\centering
  	\begin{tikzpicture}
    		\begin{axis}[step,xmax=0.06]
    			\addplot+[thick] table[mark=none,x index=0, y index=1, col sep=comma] {./csv/BP_1_step.csv}; 
    		\end{axis}
	\end{tikzpicture}
 	 \caption{Réponse Indicielle : $m=3$}
  	\label{fig:sub1}
	\end{subfigure}
	\vskip 1em
	\begin{subfigure}{.5\textwidth}
  	\centering
  	\begin{tikzpicture}
    		\begin{loglogaxis}[freqresp,xmin=10,xmax=100000]
    			\addplot+[thick] table[mark=none,x index=0, y index=1, col sep=comma] {./csv/BP_0_freqresp.csv}; 
    		\end{loglogaxis}
	\end{tikzpicture}
 	 \caption{Module de la Réponse Fréquentielle :  $m=0.1$}
  	\label{fig:sub1}
	\end{subfigure}\hfill
	\begin{subfigure}{.5\textwidth}
  	\centering
  	\begin{tikzpicture}
    		\begin{loglogaxis}[freqresp,xmin=100,xmax=10000]
    			\addplot+[thick] table[mark=none,x index=0, y index=1, col sep=comma] {./csv/BP_1_freqresp.csv}; 
    		\end{loglogaxis}
	\end{tikzpicture}
 	 \caption{Module de la Réponse Fréquentielle : $m=3$}
  	\label{fig:sub1}
	\end{subfigure}
	\vskip 1em
	\begin{subfigure}{.5\textwidth}
  	\centering
  	\begin{tikzpicture}
    		\begin{semilogxaxis}[freqresp,xmin=10,xmax=100000]
    			\addplot+[thick] table[mark=none,x index=0, y index=2, col sep=comma] {./csv/BP_0_freqresp.csv}; 
    		\end{semilogxaxis}
	\end{tikzpicture}
 	 \caption{Phase de la Réponse Fréquentielle :  $m=0.1$}
  	\label{fig:sub1}
	\end{subfigure}\hfill
	\begin{subfigure}{.5\textwidth}
  	\centering
  	\begin{tikzpicture}
    		\begin{semilogxaxis}[freqresp,xmin=100,xmax=10000]
    			\addplot+[thick] table[mark=none,x index=0, y index=2, col sep=comma] {./csv/BP_1_freqresp.csv}; 
    		\end{semilogxaxis}
	\end{tikzpicture}
 	 \caption{Phase de la Réponse Fréquentielle :  $m=3$}
  	\label{fig:sub1}
	\end{subfigure}
\caption{Analyse de deux filtres passe-bande de second ordre ($T_{m}=2$, $\omega_0=1000$ rad/s). Les figures de gauche correspondent au cas où $m=0.1$ et les figures de droite au cas où $m=3$.}
\label{fig:test}
\end{figure}

\chapter{Notch}

\begin{figure}
\centering
	\begin{subfigure}{.5\textwidth}
  	\centering
  	\begin{tikzpicture}
    		\begin{axis}[impulse,xmax=0.04]
    			\addplot+[thick] table[mark=none,x index=0, y index=1, col sep=comma] {./csv/Notch_0_impulse.csv}; 
    		\end{axis}
	\end{tikzpicture}
 	 \caption{Réponse Impulsionnelle :  $m=0.1$}
  	\label{fig:sub1}
	\end{subfigure}\hfill
	\begin{subfigure}{.5\textwidth}
  	\centering
  	\begin{tikzpicture}
    		\begin{axis}[impulse,xmax=0.06]
    			\addplot+[thick] table[mark=none,x index=0, y index=1, col sep=comma] {./csv/Notch_1_impulse.csv}; 
    		\end{axis}
	\end{tikzpicture}
 	 \caption{Réponse Impulsionnelle : $m=3$}
  	\label{fig:sub1}
	\end{subfigure}
	\vskip 1em
	\begin{subfigure}{.5\textwidth}
  	\centering
  	\begin{tikzpicture}
    		\begin{axis}[step,xmax=0.04]
    			\addplot+[thick] table[mark=none,x index=0, y index=1, col sep=comma] {./csv/Notch_0_step.csv}; 
    		\end{axis}
	\end{tikzpicture}
 	 \caption{Réponse Indicielle : $m=0.1$}
  	\label{fig:sub1}
	\end{subfigure}\hfill
	\begin{subfigure}{.5\textwidth}
  	\centering
  	\begin{tikzpicture}
    		\begin{axis}[step,xmax=0.06]
    			\addplot+[thick] table[mark=none,x index=0, y index=1, col sep=comma] {./csv/Notch_1_step.csv}; 
    		\end{axis}
	\end{tikzpicture}
 	 \caption{Réponse Indicielle : $m=3$}
  	\label{fig:sub1}
	\end{subfigure}
	\vskip 1em
	\begin{subfigure}{.5\textwidth}
  	\centering
  	\begin{tikzpicture}
    		\begin{loglogaxis}[freqresp,xmin=10,xmax=100000]
    			\addplot+[thick] table[mark=none,x index=0, y index=1, col sep=comma] {./csv/Notch_0_freqresp.csv}; 
    		\end{loglogaxis}
	\end{tikzpicture}
 	 \caption{Module de la Réponse Fréquentielle :  $m=0.1$}
  	\label{fig:sub1}
	\end{subfigure}\hfill
	\begin{subfigure}{.5\textwidth}
  	\centering
  	\begin{tikzpicture}
    		\begin{loglogaxis}[freqresp,xmin=100,xmax=10000]
    			\addplot+[thick] table[mark=none,x index=0, y index=1, col sep=comma] {./csv/Notch_1_freqresp.csv}; 
    		\end{loglogaxis}
	\end{tikzpicture}
 	 \caption{Module de la Réponse Fréquentielle : $m=3$}
  	\label{fig:sub1}
	\end{subfigure}
	\vskip 1em
	\begin{subfigure}{.5\textwidth}
  	\centering
  	\begin{tikzpicture}
    		\begin{semilogxaxis}[freqresp,xmin=10,xmax=100000]
    			\addplot+[thick] table[mark=none,x index=0, y index=2, col sep=comma] {./csv/Notch_0_freqresp.csv}; 
    		\end{semilogxaxis}
	\end{tikzpicture}
 	 \caption{Phase de la Réponse Fréquentielle :  $m=0.1$}
  	\label{fig:sub1}
	\end{subfigure}\hfill
	\begin{subfigure}{.5\textwidth}
  	\centering
  	\begin{tikzpicture}
    		\begin{semilogxaxis}[freqresp,xmin=100,xmax=10000]
    			\addplot+[thick] table[mark=none,x index=0, y index=2, col sep=comma] {./csv/Notch_1_freqresp.csv}; 
    		\end{semilogxaxis}
	\end{tikzpicture}
 	 \caption{Phase de la Réponse Fréquentielle :  $m=3$}
  	\label{fig:sub1}
	\end{subfigure}
\caption{Analyse de deux filtres rejecteur de second ordre ($T_{0}=2$, $\omega_0=1000$ rad/s). Les figures de gauche correspondent au cas où $m=0.1$ et les figures de droite au cas où $m=3$.}
\label{fig:test}
\end{figure}

\end{document}

